\documentclass[UTF8]{ctexart}
\title{操作系统实验项目}
\author{郑戈涵 17338233}
\date{\today}
\usepackage{amsmath}
\usepackage{amsfonts}
\usepackage{amssymb}
\usepackage{color}
\usepackage{enumerate}
\usepackage{graphicx}
\usepackage{ctex}
\usepackage{float}
\usepackage{shadow}
\usepackage{fancybox}
\usepackage{latexsym}
\renewcommand\figurename{图}
\renewcommand\tablename{表}
\begin{document}

\maketitle
\begin{abstract}
	本实验将完成两个任务,分别是搭建和应用实验环境和接管裸机的控制权
\end{abstract}
\tableofcontents
\section{实验要求}
\subsection{搭建和应用实验环境}
	虚拟机安装,生成一个基本配置的虚拟机XXXPC和多个1.44MB容量的虚拟软盘,将其中一个虚拟软盘用DOS格式化为DOS引导盘,用WinHex工具将其中一个虚拟软盘的首扇区填满你的个人信息。

\subsection{接管裸机的控制权}
	设计IBM_PC的一个引导扇区程序,程序功能是:用字符‘A’从屏幕左边某行位置45度角下斜射出,保持一个可观察的适当速度直线运动,碰到屏幕的边后产生反射,改变方向运动,如此类推,不断运动;在此基础上,增加你的个性扩展,如同时控制两个运动的轨迹,或炫酷动态变色,个性画面,如此等等,自由不限。还要在屏幕某个区域特别的方式显示你的学号姓名等个人信息。将这个程序的机器码放进放进第三张虚拟软盘的首扇区,并用此软盘引导你的XXXPC,直到成功。\par


\end{document}





